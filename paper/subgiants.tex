% \documentclass[useAMS, usenatbib]{emulateapj}
% \documentclass[iop]{emulateapj}
\documentclass[useAMS, usenatbib]{aastex}
\usepackage{cite,natbib}
\usepackage{epsfig}
\usepackage{cases}
\usepackage[section]{placeins}
\usepackage{graphicx, subfigure}

\newcommand{\integer}{5}
\newcommand{\nsubs}{7}
\newcommand{\dt}{$\Delta t$}

\title{Searching for planets with subgiant hosts}

\author{Ruth Angus}
\affil{Department of Physics, University of Oxford, UK}
\author{John Asher Johnson}
\affil{Harvard-Smithsonian Center for Astrophysics, Cambridge,
MA, USA}

\begin{document}

\date{Draft version 2014 December 4}
\maketitle

% \begin{abstract}

In the search for exoplanets orbiting intermediate-mass stars, focus has
shifted towards subgiants due to their slower rotation and cooler atmospheres
\citep{Johnson2014}.
To date, no planet less than 0.61 M$_{\mathrm{Jup}}$ orbiting a star of mass
$>$ 1.5 M$_\odot$ has been discovered, however the {\it Kepler} satellite has
revealed a large population of Neptunes and super-Earths orbiting less massive
stars.
Revealing such a population orbiting intermediate-mass stars could be possible
with state-of-the-art spectrographs and an optimised observing strategy.
Based on {\it Kepler} photometry and RV observations, radial velocity jitter
in subgiants is dominated by p-mode oscillations which have, of order,
hour-long characteristic timescales.
The aim of this paper is to design an observing strategy that best mitigates
the effects of stochastic variability produced by p-mode oscillations.

% \end{abstract}

\section{Introduction}
\label{intro}

% Executive summary
To understand the population of exoplanets as a function of host mass, it is
necessary to characterize the distribution of planets orbiting stars more
massive than the Sun.
To date, we know of just one Jupiter-mass planet orbiting a `retired A star'
within 0.5 AU.
Otherwise, no planet orbiting within 0.5 AU, and no exoplanet less than 0.61
M$_{Jup}$ orbiting a star of mass $>$1.5M$_{\odot}$ is known to exist.
However, because the number of planets per star increases with decreasing
planet mass for Solar-type hosts it follows that there may exist a large,
undetected population of Neptunes and super-Earths orbiting intermediate-mass
stars.

With the number of confirmed exoplanets now well into the 1500s, the new
challenge in exoplanet research lies in characterizing the population of
planets in our galaxy.
A huge effort is being put behind understanding the mass, radius and period
distributions of exoplanets orbiting Sun-like stars
\citep[e.g.][]{Butler2006, Howard2010, Petigura2013, Foreman-Mackey2014},
M dwarfs \citep{Dressing2015} and A-type stars \citep{Johnson2010a}.
Despite the need to study exoplanet populations across the full range of host
masses, focus has mostly been directed towards Sun-like and late-type stars.
This is in part due to the intrinsic difficulties faced when measuring precise
radial velocities (RVs) of hot stars.
Specifically, the spectra of hot stars are usually poorly populated with
absorption lines and, due to their rapid rotation, those lines are often very
broad.
As they evolve off the MS however, hot stars cool, expand and spin more
slowly, making them excellent RV targets: their spectra become richly
populated with narrow absorption lines and it becomes possible to measure RVs
with ms$^{-1}$ precision.

Previous studies show that there exists a positive correlation between the
number of Jupiters per star and host mass \citep{Johnson2010a, Bowler2010,
Lovis2007}.
Despite the growing understanding of planet populations for Solar-mass stars
and below, due to the paucity of detected low-mass planets orbiting massive
stars, little is understood about the distribution of planet masses across the
full stellar mass range.
Exploring the relation between planets and their hosts underpins our
understanding of planet formation mechanisms and it is essential that this gap
is filled.
Figure \ref{fig:mstar_msini} shows minimum masses of all the confirmed planets
with $m\sin i <$ 100 M$_{Earth}$ as a function of stellar mass.
The region of parameter space targeted in this study is highlighted in red.
We postulate that the lack of planets inside the red region is a selection
effect and does not represent the true distribution of exoplanets orbiting
massive planets.
Despite the increasing exoplanet occurrence rate with decreasing radius
observed for Solar-type stars, the smallest A star planet detected is 0.61
M$_{Jup}$ and no planet has been detected orbiting within 0.6 AU of an A star
\citep{Johnson2007}.
This may be due to inner planets being engulfed or their orbits
disrupted - the cause is still unclear, however a population of Neptunes and
super-Earths orbiting within 0.6 AU may have escaped detection.
We are developing a new technique for modeling RV jitter in subgiants that will
reveal such a population, if it exists.

\begin{figure*}
\begin{center}
\includegraphics[width=6in, clip=true]{mstar_msini.pdf}
\caption{Host star mass vs minimum masses for planets with orbital solutions
listed in the {\it Exoplanet.org}. In this study we are targeting the blue
region.}
\label{fig:mstar_msini}
\end{center}
\end{figure*}

With the incredible precision achievable with modern spectrographs, the
limiting factor in RV exoplanet searches is now astrophysical, not
instrumental noise.
In Solar and low-mass main sequence stars, the dominant contribution to RV
jitter is produced by inhomogeneous surface features; spots and plages, which
periodically mask regions of the approaching and receding stellar hemispheres \citep{Aigrain2012, Dumusque2011, Haywood2014}.
In subgiants however, p-mode oscillations are the dominant noise
source, producing RV jitter with a typical amplitude of 5 ms$^{-1}$, on
timescales of a few hours.
When conditioning a Keplerian orbit model on a noisy RV dataset it is
customary to inflate instrumental error bars by some jitter term. In the
presence of white noise this may be the best approach, however typical RV
jitter is correlated on timescales of a few hours.
% Additionally, RV jitter in Subgiants has an astrophysical origin and
% can therefore be modeled.
\citep{Wright2005} showed that RV jitter increases with evolutionary stage,
finding a median jitter level of around 5.7 ms$^{-1}$ for subgiants and around
4 ms$^{-1}$ for main-sequence stars.
The most probable cause of this increased jitter, p-mode oscillations,
are apparent in the two epochs of KECK RV observations of
HD 142091 shown in figures \ref{fig:k2_1} and \ref{fig:k2_2}.
A linear combination of sine waves, with frequencies calculated using the
scaling relations in \citep{Kjeldsen1995} were fit to both sets of
observations, shown in blue.
The two global asteroseismic parameters, $\delta \nu$ and $\nu_{max}$ were
estimated using these scaling relations and mass, radius and T$_{\mathrm{eff}}$ measurements from (REFERENCE).
The frequencies of the sine waves were then calculated as $\nu_{max}-n\delta
\nu$, $\nu_{max}-(n-1)\delta \nu$, ..., for $n$=3 frequencies.

\begin{figure*}
\begin{center}
\includegraphics[width=6in, clip=true]{142091_1.pdf}
\caption{KECK RV observations of HD 142091. A fit to the data using a linear
combination of 3 theoretical oscillation frequencies is shown in blue. The RMS
of the residuals is 1.36 ms$^{-1}$.}
\label{fig:k2_1}
\end{center}
\end{figure*}

\begin{figure*}
\begin{center}
\includegraphics[width=6in, clip=true]{142091_2.pdf}
\caption{KECK RV observations of HD 142091. A fit to the data using a linear
combination of 3 theoretical oscillation frequencies is shown in blue. The RMS
of the residuals is 1.55 ms$^{-1}$.}
\label{fig:k2_2}
\end{center}
\end{figure*}

In section \textsection \ref{section:method} of this paper we use the
assumption that p-mode oscillations provide the dominant contribution to RV
jitter in subgiants to produce a simulated RV data set and assess the impact
of different observing strategies on our ability to detect planets.
We test the impact of our optimised observing strategy on planet detection
limits in section \textsection \ref{section:detection}.

% The giant star, $\beta$ Geminorum has long been known to exhibit
% asteroseismic oscillations that are detectable with RVs.
% \citep{Hatzes2012} measured the mass of $\beta$ Gem using both photometric
% and spectroscopic observations.

\subsection{An optimum observing strategy: analysis based on HD189351}
\label{section:method}

In order to find the best observing method that reduces the effects of p-mode
oscillations on RV observations of subgiants it was first necessary to develop
a method of generating synthetic data.
Asteroseismic pulsations produce periodic luminosity fluctuations, detectable
in short cadence light curves due to the excellent precision of
{\it Kepler}.
The variation in brightness is brought about by a change in the star's surface
temperature: the stellar envelope cools and decreases in luminosity as the
surface layers move away from the centre of the star and gets brighter again
as the surface moves closer to the stellar centre.
These surface variations do not only cause luminosity fluctuations, they also
produce variations in radial velocity.
Equation 5 in \citet{Kjeldsen1995} relates the fractional variation in
bolometric luminosity to radial velocity:
\begin{equation}
	\delta L/L_{bol} = \frac{v_{osc}/\mathrm{ms}^{-1}}
	{T_{\mathrm{eff}}/5777\mathrm{K}}17.7\mathrm{ppm},
\end{equation}
\label{eq:flux_rv}
where ppm denotes parts-per-million.
We obtained {\it Kepler} short-cadence PDC-MAP light curves
\citep[][]{Smith2012, Stumpe2012} of HD185351, a subgiant that was analysed in
detail by \citet{Johnson2014} and used equation \ref{eq:flux_rv} to transform
the light curves into RV curves.
We then fitted a linear combination of sine waves to the time series, using
the 6 highest amplitude oscillation frequencies reported in
\citet{Johnson2014}, ranging from 218 to 250 $\mu$Hz (period$\sim$2.7-3.2
hours).
The resulting amplitudes of each sine wave were then used to generate
synthetic data, at arbitrary observation times, on which we could train our
observing strategy.
The synthetic RV curve for HD185351 is shown in the top panel of figure
\ref{fig:os}.

In order to establish an optimum observing strategy we attempted to minimise
the RMS of simulated RV observations with respect to \dt, the time between
intra-night observations.
Since p-mode oscillations affect subgiant RVs on roughly hour-long timescales,
this variability will not in generaal be averaged out over one exposure, as
might be the case for main sequence stars.
It is therefore necessary to develop an observing strategy for these stars.
Using the simulated RV curve of HD185351, we recorded the RV once per night,
integrated over a 100 second exposure time, at the same time each night for 10 consecutive nights.
The RMS of these 10 RV observations was calculated.
Then, three observations 2 minutes apart were made each night, again at the
same time each night for 10 consecutive nights.
The mean of these three measurements was taken and the RMS of the resulting 10
mean values calculated.
We increased \dt$~$from 2 to 200, calculating the RMS of RV observations
each time.
Since the time at which the first observation was made, the `starting time',
$t_0$, was chosen arbitrarily, we marginalised over this parameter by
calculating RMS as a function of \dt, for a range of $t_0$ and calculating the
mean.
The variation in RMS as a function of \dt$~$is shown in figure \ref{fig:os}.
The top panel shows the synthetic 10 day RV curve for HD185351, a 1.99$\pm$0.23
M$_\odot$, 5.35$\pm$0.20 R$_\odot$ subgiant.
The bottom panel shows the variation in RMS with \dt.
Faint orange lines show the variation in RMS as a function of \dt$~$for a range
of $t_0$s and the solid orange line shows the mean of these lines.
The solid orange line shows the variation of RMS with \dt, marginalised over
$t_0$.

\begin{figure*}
\begin{center}
\includegraphics[width=6in, clip=true]{HD185_3_10_hd_os.pdf}
\caption{{\it (Top)} Synthetic 10 day RV curve for HD185351, a 1.99$\pm$0.23
M$_\odot$, 5.35$\pm$0.20 R$_\odot$ subgiant. {\it(Bottom)} RMS vs \dt. The
faint orange lines show the variation in RMS as a function of \dt$~$for a range
of starting times. The solid orange line shows the mean of these lines. The
solid orange line shows the effect of marginalising over the starting times.}
\label{fig:os}
\end{center}
\end{figure*}

\subsection{Results of initial PFS observations}

Based on the simulations described above, we selected two stars with similar
properties to HD185351, that were observable from Las Campanas observatory
during January and February of 2015.
These stars were selected from a sample of $\sim$ 245 subgiants with
spectroscopic stellar parameters, from (CITATION).
For each star in this catalogue we calculated a value of $\nu_{max}$,
using the asteroseismic scaling relations and compared this value to that of
HD185351.
Based on these criteria, we selected HD95089 and HD98219, for an initial
observing run on the Planet Finder Spectrograph (PFS) instruments on the
Magellan telescope.
The properties of these two stars and HD185351 are listed in table
\ref{tab:subs}.

\begin{table*}
\caption{Spectroscopic and theoretical stellar parameters of HD185351 and the
two stars targeted in the initial PFS campaign.}
\begin{tabular}{lccccc}
\hline\hline
Star & M (M$_\odot$) & R R($_\odot$) & T$_{\mathrm{eff}}$(K)
& $\Delta \nu$($\mu$Hz) & $\nu_{max}$($\mu$Hz) \\
    \hline
    HD185351 & 1.99$\pm$0.23 & 5.35$\pm$0.20 & 5016$\pm$44 & 15.38 & 227.57 \\
HD95089  & 1.59863 & 4.55568 & 4974.45 & 17.54 & 253.17 \\
HD98219  & 1.29773 & 4.33277 & 4913.12 & 17.04 & 228.63 \\
\hline
\end{tabular}
\label{tab:subs}
\end{table*}

... RV measurements over ... nights were obtained.
These RVs are shown in figures \ref{fig:HD95089} and \ref{fig:HD98219}.

\begin{figure*}
\begin{center}
\includegraphics[width=6in, clip=true]{HD95089.pdf}
\caption{RV observations of the subgiant HD95089.}
\label{fig:HD95089}
\end{center}
\end{figure*}

\begin{figure*}
\begin{center}
\includegraphics[width=6in, clip=true]{HD98219.pdf}
\caption{RV observations of the subgiant HD98219.}
\label{fig:HD98219}
\end{center}
\end{figure*}

\subsection{An optimum observing strategy: generalising to arbitrary mass and
radius}

We obtained {\it Kepler} short-cadence PDC-MAP light curves of a further
\nsubs$~$ subgiants or `mixed mode' stars with a range of masses and radii and
individual oscillation mode measurements published in \citet{Appourchaux2012}.

% These {\it Kepler} subgiants have masses ranging from 1.0 to 1.6 M$_\odot$ and
There exist several {\it Kepler} subgiants whose individual oscillation
frequencies have been detected, however their masses range from 1.0 to 1.6
M$_\odot$ and their radii from 1.9 to 2.8 R$_\odot$.
We are interested in stars with radii up to at least 10 R$_\odot$.
It was therefore necessary to produce sythetic RV curves for larger stars.
We have precise spectroscopic estimates of mass, radius and temperature for
245 subgiants in the Hipparcos catalogue with masses ranging from 0.9 to 2.6
M$_\odot$ and radii from 1.7 to 16 R$_\odot$.
Using these masses, radii and effective temperatures, we generated RV curves
with frequencies calculated using the asteroseismic scaling relations.
An example of one of these synthetic light curves is shown in figure
\ref{fig:hd1100_rvs}.
Theoretical oscillation frequencies, calculated using the stellar parameters
of HD1100 were used to generate this RV curve.
The blue regions shows the area represented by an uncertainty of 2ms$^{-1}$.
The bottom panel shows the periodogram of the top RV curve.
The peaks have finite width due to the finite length of the RV curve.

\begin{figure*}
\begin{center}
\includegraphics[width=6in, clip=true]{hd1100_rvs.pdf}
\caption{{\it (Top)} Simulated RV curve generated using 12
theoretical oscillation frequencies, based on the stellar parameters of HD1100.
The blue region shows the area represented by an uncertainty of 2ms$^{-1}$.
{\it (Bottom)} The periodogram of the top RV curve. The peaks have a finite
width due to the finite length of the time series.}
\label{fig:hd1100_rvs}
\end{center}
\end{figure*}

% We compared the two methods of generating synthetic RVs: one that involves
% training on {\it Kepler} light curves and one that uses the scaling relations.
% Figure \ref{fig:hd1100_rvs} shows a synthetic RV curve, generated using 12
% theoretical oscillation frequencies with the stellar parameters of HD1100.
% The blue regions shows the area represented by an uncertainty of 2ms$^{-1}$.
% The bottom panel shows the periodogram of the top RV curve.
% The peaks have finite width due to the finite length of the RV curve.

We repeated  this analysis for each subgiant in our catalogue, finding the
\dt$~$that minimises the RMS in each case.
The resulting \dt$~$as a function of stellar radius is shown in figure
\ref{fig:r_vs_time}.
The blue line is a 2D polynomial fit to the data.
% Figure \ref{fig:AMP_r_vs_time} shows radius vs \dt for the {\it Kepler} AMP
% subgiants, i.e., the subgiants with individual oscillation frequencies.
The synthetic RV curves for these stars were trained on the light curves.
These stars have much smaller radii and therefore their timescales of
oscillation are much shorter.
For this reason, integration times of 100 seconds average out most of the
variation and developing an optimal observing strategy is not necessary for
these stars.
Amongst these stars there is also a clear trend between \dt and radius.
When scheduling RV observations of subgiants for exoplanet search, the interval
between intra-night observations, \dt$~$depends on the star's radius, mass and
effective temperature.
This is exactly the result one would expect given the fact that these
3 properties of a star govern the timescale of p-mode oscillations.

\begin{figure*}
\begin{center}
\includegraphics[width=6in, clip=true]{r_vs_time.pdf}
\caption{Radius vs \dt. This plot shows how the optimal observing interval,
\dt, varies as a function of subgiant radius. The blue line is a 2D polynomial
fit to the data.}
\label{fig:r_vs_time}
\end{center}
\end{figure*}

% \begin{figure*}
% \begin{center}
% \includegraphics[width=6in, clip=true]{AMP_r_vs_time.pdf}
% \caption{Radius vs \dt for the AMP subgiants. This plot shows how the optimal
% observing interval, \dt, varies as a function of subgiant radius.}
% \label{fig:AMP_r_vs_time}
% \end{center}
% \end{figure*}

\section{Detection limits}
\label{section:detection}

In order to assess our ability to recover planets we injected signals into
synthetic RV curves and attempted to recover them using our optimised observing
strategy.
We generated 1000 planets with masses ranging from 10-100 M$_{Earth}$ and
periods ranging from 2-20 days, drawing randomly from log-uniform
distributions in the two parameters.
We then injected these planet signals into 1000 RV curves with white noise,
generated from the 100 short-cadence targets used in section \ref{sub:optimum}.
We found that the detection threshold is reduced...

\section{Results and Discussion}
\label{results}

\section{Conclusion}
\label{conclusion}

\bibliographystyle{plainnat}
\bibliography{subgiants}

\end{document}

% Intro:
% Wright, Saar
% Showing noise source = oscillation modes
% Keck observations, show that scatter is close to instrumental precision
% also do it for Beta Gem
% granulation and oscillations operate on different timescales
% talk about Kepler results
% How to generate time series that looks realistic?
% sum of sine waves
% gaussian process
% Observing strategy
% Johnson et al 2011, reproduce figure
% Do it for a giant star
% Hekker and Raffert, excess scatter as a function of logg
% possible aliasing
% detection limits

% To do next
% intro
% make figure with k2 target data
% write about the history of giants and RVs (Beta Gem and hd...)
% make a plot of Beta Gem data
% write about Kepler data

% method
% write about method
% talk a little about Xavier's paper
% make observing strategy code more general
% Convert flux to RVs using kjeldsen relation.
% Use the Kepler target john sent
% Use a 10 day segment
% train a GP / train a sum of sine waves

% detection limits
% read about what Xavier did
% inject planets using the expected distribution
% try to recover them
