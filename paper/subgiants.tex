% \documentclass[useAMS, usenatbib]{emulateapj}
% \documentclass[iop]{emulateapj}
\documentclass[useAMS, usenatbib]{aastex}
\usepackage{cite,natbib}
\usepackage{epsfig}
\usepackage{cases}
\usepackage[section]{placeins}
\usepackage{graphicx, subfigure}

\title{Searching for planets with subgiant hosts}

\author{Ruth Angus}
\affil{Department of Physics, University of Oxford, UK}
\author{John Asher Johnson}
\affil{Harvard-Smithsonian Center for Astrophysics, Cambridge,
MA, USA}

\begin{document}

\date{Draft version 2014 December 4}
\maketitle

% \begin{abstract}

% Context:
Searches for extrasolar planets around massive stars concentrate on
the subgiants due to their slower rotation and cooler atmospheres
\citep{Johnson2014}.
The greater mass of these targets make the precision of RV measurements more
important than ever.
It is therefore necessary to devise an optimal observing strategy for subgiant
stars which have different RV jitter producing processes to Sun-like stars.
% Aims:
The aim of this paper is to simulate the behaviour of subgiant RV
measurements and design an observing strategy that best mitigates the effects
of stochastic variability produced by p-mode oscillations and granulation.
% Method:
We train a Gaussian process on existing data to reproduce realistic
subgiant RVs.
We then test a range of different observing strategies in order
to find the optimum observing strategy that best reduces the impact of
RV jitter.
We also use simulations to make predictions about the minumum masses and radii
of planets that may be detectable with current technology.
% Results:
% We find that...

% \end{abstract}

\section{Introduction}
\label{intro}

% Executive summary
To understand the population of exoplanets as a function of host mass, it is
essential that we characterize the distribution of planets orbiting massive
stars.
To date, we know of just one Jupiter-mass planet orbiting a retired A star
within 0.5 AU.
Otherwise, no planet orbiting within 0.5 AU, and no exoplanet less than 0.61
M$_Jup$ orbiting a star of mass $>$1.5M$_{\odot}$ is known to exist.
However, because the number of planets per star increases with decreasing
planet mass for Solar-type hosts it follows that there may exist a large,
undetected population of Neptunes and super-Earths orbiting massive stars.

% This paucity is somewhat surprising because the number of
% planets per star increases with decreasing planet mass for Solar-type hosts.
% It follows that there may exist a large, undetected population of Neptunes
% and super-Earths orbiting massive stars.

% By observing evolved A-stars it is possible to overcome difficulties in
% measuring precise radial velocities (RVs) for massive stars, however, planet
% signals may be been buried by the ~5ms-1 RV jitter produced by p-mode
% oscillations in subgiants.

With the number of confirmed exoplanets now well into the 1500s, the new
challenge in exoplanet research lies in characterizing the population of
planets in our galaxy.
A huge effort is being made to understand the mass, radius and period
distributions of exoplanets orbiting Sun-like stars
\citep[e.g.][]{Butler2006, Howard2010, Petigura2013, Foreman-Mackey2014},
M dwarfs (Dressing et al., in prep; Gaidos et al., 2014) and A-type stars
\citep{Johnson2010a}.
Despite the need to study exoplanet populations across the full range of host
masses, focus has mostly been directed towards Sun-like and late-type stars.
This is in part due to the intrinsic difficulties faced in measuring precise
radial velocities (RVs) of hot stars.
Specifically, the spectra of hot stars are usually poorly populated with
absorption lines and, due to their rapid rotation, those lines are often very
broad.
As they evolve off the MS however, hot stars cool, expand and spin more
slowly, making them excellent RV targets: their spectra become richly
populated with narrow absorption lines and it is possible to measure RVs with
ms$^-1$ precision.

Previous studies show that there exists a positive correlation between the
number of Jupiters per star and host mass \citep{Johnson2010a, Bowler2010,
Lovis2007}.
Despite the growing understanding of planet occurrence rates for Solar-mass
stars and below, due to the paucity of
detected low-mass planets orbiting massive stars, little is understood about
the distribution of planet masses across the full stellar mass range.
Understanding the relation between planets and their hosts is absolutely key
to our understanding of planet formation mechanisms---it is essential that
this gap is filled.
% Figure 1 shows minimum masses of all the confirmed planets
% with $m\sin i <$ 100 M$_{Earth}$ as a function of stellar mass.
The region of parameter space targeted in this study is highlighted in red.
We postulate that the lack of planets inside the red region is a selection
effect and does not represent the true distribution of exoplanets orbiting
massive planets.
Despite the increasing exoplanet occurrence rate with decreasing radius
observed for Solar-type stars, the smallest A star planet detected is 0.61 Mjup
and no planet has been detected orbiting within 0.6 AU of an A star
\citep{Johnson2007}.
This may be due to inner planets being engulfed or their orbits
disrupted - the cause is still unclear, however a population of Neptunes and
super-Earths orbiting within 0.6 AU may have escaped detection.
We are developing a new technique for modeling RV jitter in subgiants that will
reveal such a population, if it exists.

With the incredible precision achievable with modern spectrographs, the
limiting factor in RV exoplanet searches is now astrophysical, not
instrumental noise.
In Solar and low-mass main sequence stars, the dominant contribution to RV
jitter is produced by inhomogeneous surface features; spots and plages, which
periodically mask regions of the approaching and receding hemispheres of the
stars \citep{Aigrain2012, Dumusque2011, Haywood2014}.
In subgiants however, p-mode oscillations are the dominant noise
source \citep{Hatzes2012}, and produce RV jitter with
a typical amplitude of 5 ms-1, on timescales of a few hours (see figure 2).
When conditioning a Keplerian orbit model on a noisy RV dataset it is
customary to inflate instrumental error bars by some jitter term. In the
presence of white noise this may be the best approach, however typical RV
jitter is correlated on timescales of a few hours.
Additionally, RV jitter in Subgiants has an astrophysical origin and
can therefore be modeled.
\citep{Hatzes2012} computed a power spectrum from photometric observations of
giant star $\beta$ Gem and then modeled the RV curve using the dominant
oscillation modes found in the power spectrum (see figure 3).
They used p-mode oscillation frequencies to calculate the stellar mass; we
intend to do the reverse.
Using knowledge of the stellar mass and radius, one can compute the
oscillation frequencies and model the radial velocity signal as a linear
combination of the dominant oscillation modes.
Modeling RV curves with this technique will enable us to reduce the typical
jitter levels to $\sim 2 ms^{-1}$.

\section{Method}
\label{method}

\section{Results and Discussion}
\label{results}

\section{Conclusion}
\label{conclusion}

\bibliographystyle{plainnat}
\bibliography{subgiants}

\end{document}
