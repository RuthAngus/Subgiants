% \documentclass[useAMS, usenatbib]{emulateapj}
% \documentclass[iop]{emulateapj}
\documentclass[useAMS, usenatbib]{aastex}
\usepackage{cite,natbib}
\usepackage{epsfig}
\usepackage{cases}
\usepackage[section]{placeins}
\usepackage{graphicx, subfigure}

\newcommand{\integer}{5}
\newcommand{\nsubs}{7}

\title{Searching for planets with subgiant hosts}

\author{Ruth Angus}
\affil{Department of Physics, University of Oxford, UK}
\author{John Asher Johnson}
\affil{Harvard-Smithsonian Center for Astrophysics, Cambridge,
MA, USA}

\begin{document}

\date{Draft version 2014 December 4}
\maketitle

% \begin{abstract}

In the search for exoplanets orbiting intermediate-mass stars, focus has
shifted towards subgiants due to their slower rotation and cooler atmospheres
\citep{Johnson2014}.
To date, no planet less than 0.61 M$_{\mathrm{Jup}}$ orbiting a star of mass
$>$ 1.5 M$_\odot$ has been discovered, however the {\it Kepler} satellite has
revealed a large population of Neptunes and super-Earths orbiting less massive
stars.
Revealing such a population orbiting intermediate-mass stars could be possible
with state-of-the-art spectrographs and an optimised observing strategy.
Radial velocity jitter in subgiants seems to be dominated by p-mode
oscillations, which have, of order, hour-long characteristic timescales.
The aim of this paper is to design an observing strategy that best mitigates
the effects of stochastic variability produced by p-mode oscillations.

% \end{abstract}

\section{Introduction}
\label{intro}

% Executive summary
To understand the population of exoplanets as a function of host mass, it is
necessary to characterize the distribution of planets orbiting massive
stars.
To date, we know of just one Jupiter-mass planet orbiting a `retired A star'
within 0.5 AU.
Otherwise, no planet orbiting within 0.5 AU, and no exoplanet less than 0.61
M$_{Jup}$ orbiting a star of mass $>$1.5M$_{\odot}$ is known to exist.
However, because the number of planets per star increases with decreasing
planet mass for Solar-type hosts it follows that there may exist a large,
undetected population of Neptunes and super-Earths orbiting massive stars.

% This paucity is somewhat surprising because the number of
% planets per star increases with decreasing planet mass for Solar-type hosts.
% It follows that there may exist a large, undetected population of Neptunes
% and super-Earths orbiting massive stars.

% By observing evolved A-stars it is possible to overcome difficulties in
% measuring precise radial velocities (RVs) for massive stars, however, planet
% signals may be been buried by the ~5ms-1 RV jitter produced by p-mode
% oscillations in subgiants.

With the number of confirmed exoplanets now well into the 1500s, the new
challenge in exoplanet research lies in characterizing the population of
planets in our galaxy.
A huge effort is being put behind understanding the mass, radius and period
distributions of exoplanets orbiting Sun-like stars
\citep[e.g.][]{Butler2006, Howard2010, Petigura2013, Foreman-Mackey2014},
M dwarfs \citep{Dressing2015} and A-type stars \citep{Johnson2010a}.
Despite the need to study exoplanet populations across the full range of host
masses, focus has mostly been directed towards Sun-like and late-type stars.
This is in part due to the intrinsic difficulties faced when measuring precise
radial velocities (RVs) of hot stars.
Specifically, the spectra of hot stars are usually poorly populated with
absorption lines and, due to their rapid rotation, those lines are often very
broad.
As they evolve off the MS however, hot stars cool, expand and spin more
slowly, making them excellent RV targets: their spectra become richly
populated with narrow absorption lines and it becomes possible to measure RVs
with ms$^{-1}$ precision.

Previous studies show that there exists a positive correlation between the
number of Jupiters per star and host mass \citep{Johnson2010a, Bowler2010,
Lovis2007}.
Despite the growing understanding of planet populations for Solar-mass stars
and below, due to the paucity of detected low-mass planets orbiting massive
stars, little is understood about the distribution of planet masses across the
full stellar mass range.
Exploring the relation between planets and their hosts underpins our
understanding of planet formation mechanisms and it is essential that this gap
is filled.
Figure \ref{fig:mstar_msini} shows minimum masses of all the confirmed planets
with $m\sin i <$ 100 M$_{Earth}$ as a function of stellar mass.
The region of parameter space targeted in this study is highlighted in red.
We postulate that the lack of planets inside the red region is a selection
effect and does not represent the true distribution of exoplanets orbiting
massive planets.
Despite the increasing exoplanet occurrence rate with decreasing radius
observed for Solar-type stars, the smallest A star planet detected is 0.61
M$_{Jup}$ and no planet has been detected orbiting within 0.6 AU of an A star
\citep{Johnson2007}.
This may be due to inner planets being engulfed or their orbits
disrupted - the cause is still unclear, however a population of Neptunes and
super-Earths orbiting within 0.6 AU may have escaped detection.
We are developing a new technique for modeling RV jitter in subgiants that will
reveal such a population, if it exists.

\begin{figure*}
\begin{center}
\includegraphics[width=6in, clip=true]{mstar_msini.pdf}
\caption{Host star mass vs minimum masses for planets with orbital solutions
listed in the {\it Exoplanet.org}. In this study we are targeting the blue
region.}
\label{fig:mstar_msini}
\end{center}
\end{figure*}

With the incredible precision achievable with modern spectrographs, the
limiting factor in RV exoplanet searches is now astrophysical, not
instrumental noise.
In Solar and low-mass main sequence stars, the dominant contribution to RV
jitter is produced by inhomogeneous surface features; spots and plages, which
periodically mask regions of the approaching and receding stellar hemispheres \citep{Aigrain2012, Dumusque2011, Haywood2014}.
In subgiants however, p-mode oscillations are the dominant noise
source, producing RV jitter with a typical amplitude of 5 ms$^{-1}$, on
timescales of a few hours.
When conditioning a Keplerian orbit model on a noisy RV dataset it is
customary to inflate instrumental error bars by some jitter term. In the
presence of white noise this may be the best approach, however typical RV
jitter is correlated on timescales of a few hours.
% Additionally, RV jitter in Subgiants has an astrophysical origin and
% can therefore be modeled.
\citep{Wright2005} showed that RV jitter increases with evolutionary stage,
finding a median jitter level of around 5.7 ms$^{-1}$ for subgiants and around
4 ms$^{-1}$ for main-sequence stars.
The most probable cause of this increased jitter, p-mode oscillations,
are apparent in the two epochs of KECK RV observations of
HD 142091 shown in figures \ref{fig:k2_1} and \ref{fig:k2_2}.
A linear combination of sine waves, with frequencies calculated using the
scaling relations in \citep{Kjeldsen1995} were fit to both sets of
observations, shown in blue.
The two global asteroseismic parameters, $\delta \nu$ and $\nu_{max}$ were
estimated using these scaling relations and mass, radius and T$_{\mathrm{eff}}$ measurements from (REFERENCE).
The frequencies of the sine waves were then calculated as $\nu_{max}-n\delta
\nu$, $\nu_{max}-(n-1)\delta \nu$, ..., for $n$=13 frequencies.
These observations provide evidence that the dominant cause of RV jitter in
subgiants is p-mode oscillations.

\begin{figure*}
\begin{center}
\includegraphics[width=6in, clip=true]{142091_1.pdf}
\caption{KECK RV observations of HD 142091. A fit to the data using a linear
combination of 13 theoretical oscillation frequencies is shown in blue. The RMS
of the residuals is 1.36 ms$^{-1}$.}
\label{fig:k2_1}
\end{center}
\end{figure*}

\begin{figure*}
\begin{center}
\includegraphics[width=6in, clip=true]{142091_2.pdf}
\caption{KECK RV observations of HD 142091. A fit to the data using a linear
combination of 13 theoretical oscillation frequencies is shown in blue. The RMS
of the residuals is 1.55 ms$^{-1}$.}
\label{fig:k2_2}
\end{center}
\end{figure*}

In section \textsection \ref{section:method} of this paper we use the
assumption that p-mode oscillations provide the dominant contribution to RV
jitter in subgiants to produce a simulated RV data set and assess the impact
of different observing strategies on our ability to detect planets.
We test the impact of our optimised observing strategy on planet detection
limits in section \textsection \ref{section:detection}.

% The giant star, $\beta$ Geminorum has long been known to exhibit
% asteroseismic oscillations that are detectable with RVs.
% \citep{Hatzes2012} measured the mass of $\beta$ Gem using both photometric
% and spectroscopic observations.

\section{Method}
\label{method}

In order to test an observing method that best reduces the effects of p-mode
oscillations on RV observations of subgiants it was first necessary to develop
a method of generating synthetic data.
We obtained {\it Kepler} short-cadence PDC-MAP light curves
\citep[][]{Smith2012, Stumpe2012} of \nsubs subgiants or `mixed mode' stars
with a range of masses and radii and individual oscillation mode measurements
published in \citet{Appourchaux2012}.
Asteroseismic pulsations produce periodic brightness fluctuations, detectable
in these short cadence light curves due to the excellent precision of
{\it Kepler}.
The variation in brightness is brought about by a change in the star's surface
temperature: the stellar envelope cools and decreases in luminosity as the
surface layers move away from the centre of the star and gets brighter again
as the surface moves closer to the stellar centre.
These surface variations do not only cause luminosity fluctuations, they also
produce variations in radial velocity.
Equation 5 in \citet{Kjeldsen1995} relates the fractional variation in
bolometric luminosity to radial velocity:
\begin{equation}
	\delta L/L_{bol} = \frac{v_{osc}/\mathrm{ms}^{-1}}
	{T_{\mathrm{eff}}/5777\mathrm{K}}17.7\mathrm{ppm},
\end{equation}
where ppm denotes parts-per-million.
We used this relation to transform the {\it Kepler} light curves into RV
curves.

Once the {\it Kepler} light curves had been converted into RV curves we
used the oscillation frequencies published in \citet{Appourchaux2012}
to fit a linear combination of sine waves to the time series.
The resulting amplitudes were then used to generate synthetic data, at
arbitrary observation times, on which we could train our observing strategy.

Compare this method to the method of just generating rv curves -
check that they give the same results!

% Using a Gaussian process (GP) with a mixture of Gaussians in Fourier space
% kernel function allowed us to realistically reproduce the p-mode contribution
% to the RV signal of subgiants.
% A GP can be defined as any stochastic process
% where the probability distribution over n observations is an n-dimensional
% Gaussian.
% Rather than parameterizing the stochastic process itself, one parameterizes the
% covariance of the data using a kernel function.
% GPs can be used as extremely flexible models for stochastic processes and have
% just a few parameters.
% As such they have recently gained popularity in astrophysics (cite Neale,
% Suzanne, Dan, Ed, raphael).
%
% Speed, choice of kernel function, description of maths, choice of kernel
% parameters.

% As well as using a GP to generate synthetic data we also used a linear
% combination of sine waves with frequencies calculated with the
% \citet{Kjeldsen1995} scaling relations and amplitudes learnt from the data.
% However, we expect the GP model to be the more realistic since oscillation
% modes frequencies are not restricted to be delta functions, rather are allowed
% to have a finite width as is reflective of true observations.

\subsection{An optimum observing strategy}
\label{sub:optimum}

In order to establish an optimum observing strategy we constructed the
following test.
The instantaneous RV was recorded once per night, at the same time each night
for 10 consecutive nights.
The RMS of these 10 observations was calculated.
Then, three instantaneous RVs per night at \integer$~$minute intervals were
recorded at the same time each night for 10 consecutive nights.
The mean of these three measurements was calculated and the RMS of the
resulting 10 mean values was calculated.
This analysis was repeated for 10 minute intervals, 15 minute intervals, etc.
The RMS as a function of time between intra-night observations, or
`observing interval' for one synthetic RV curve is shown in figure ..
The absolute time at which the RVs were recorded was arbitrary, however since
different choices of the starting observing time results in different
calculated RMS values due to the stochastic nature of the RV time series.
To combat this, we performed the same analysis at a range of
starting times, spaced by \integer$~$minute intervals.
The resulting RMS curves are over-plotted in the bottom panel of figure ...
as faint orange lines.

\section{Detection limits}
\label{section:detection}

In order to assess our ability to recover planets we injected signals into
synthetic RV curves and attempted to recover them using our optimised observing
strategy.
We generated 1000 planets with masses ranging from 10-100 M$_{Earth}$ and
periods ranging from 2-20 days, drawing randomly from log-uniform
distributions in the two parameters.
We then injected these planet signals into 1000 RV curves with white noise,
generated from the 100 short-cadence targets used in section \ref{sub:optimum}.
We found that the detection threshold is reduced...

\section{Results and Discussion}
\label{results}

\section{Conclusion}
\label{conclusion}

\bibliographystyle{plainnat}
\bibliography{subgiants}

\end{document}

% Intro:
% Wright, Saar
% Showing noise source = oscillation modes
% Keck observations, show that scatter is close to instrumental precision
% also do it for Beta Gem
% granulation and oscillations operate on different timescales
% talk about Kepler results
% How to generate time series that looks realistic?
% sum of sine waves
% gaussian process
% Observing strategy
% Johnson et al 2011, reproduce figure
% Do it for a giant star
% Hekker and Raffert, excess scatter as a function of logg
% possible aliasing
% detection limits

% To do next
% intro
% make figure with k2 target data
% write about the history of giants and RVs (Beta Gem and hd...)
% make a plot of Beta Gem data
% write about Kepler data

% method
% write about method
% talk a little about Xavier's paper
% make observing strategy code more general
% Convert flux to RVs using kjeldsen relation.
% Use the Kepler target john sent
% Use a 10 day segment
% train a GP / train a sum of sine waves

% detection limits
% read about what Xavier did
% inject planets using the expected distribution
% try to recover them
